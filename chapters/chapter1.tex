\chapter{Introduction}%
\label{chapter:introduction}

\begin{introduction}
Face your life, its pain, its pleasure, leave no path untaken. - Neil Gaiman
\end{introduction}


\section{Motivation}
Pain is unpleasant. No one enjoys experiencing it, whether it's the muscle soreness from working out or the more severe pain from a broken bone. In fact, it might even represent a significant burden on a person's quality of life. The unpleasant feeling it provokes may result in functional disability or even a change of behaviour, making an individual feel anger and frustration \cite{Dirk2021}. However, pain serves a valuable purpose: it alerts us to potential dangers and helps us adapt to situations that could harm us, helping us avoid further injuries \cite{Coninx2021}. Nevertheless, when it interferes with the way people typically live their lives, it's important to find ways to attenuate it. This burden is particularly profound for patients suffering from chronic pain conditions, for whom discomfort becomes a persistent and debilitating element of life.

Attempts of measuring pain surged with the intent of improving quality of pain management \cite{Nugent2021}. For this reason, scales like the \ac{nps} and the \ac{vas} have been used since 1950 \cite{Bielewicz2022}. Although these are the most used methods, they're very subjective, depending on a patient's own perspective of pain. Moreover, some patients might undermine the severity of their pain or be unwilling to share their experience. For those with cognitive difficulties, these methods of verbal report may not even be appropriate, especially for non-verbal individuals \cite{Qin2022}.

The challenge of measuring pain makes prescribing pain medication a difficult job for healthcare providers since the only comparison they have is a patient's previous results. This means that the prescription can be excessive and the organism will grow accustomed to it quicker or, if the patient minimizes the severity, they will continue to experience pain as their tolerance increases. This highlights the importance of finding an objective way of describing pain.





\section{Work Objectives}









\section{Pain}

According to its duration, pain can be classified as acute or chronic. Acute pain is induced by the activation of nociceptor sensory neurons, which occurs in the presence of actual or potential damaging stimuli, such as intense heat or cold and excessive mechanical force, or due to inflammation \cite{Jayakar2021}. On the other hand, chronic pain is defined as lasting more than three months \cite{Raman2022} and can be classified into nociceptive, neuropathic or nociplastic pain. Nociceptive pain results from continuous stimuli associated with tissue injury, while neuropathic pain results from damage to the peripheral or central nervous system. Lastly, nociplastic pain is a broader term, that is applied to chronic pain when it can't be described by the other two terms \cite{Fitzcharles2021}.

The Numerical Pain Scale is one of the most widely used traditional methods for assessing pain. It typically involves asking patients to rate their pain on a scale from 0 to 10, where 0 represents no pain and 10 signifies the worst pain imaginable \cite{Nugent2021}. While simple and easy to administer, this method relies entirely on the individual’s subjective interpretation of their pain, which can be influenced by factors such as mood, stress and gender. Besides this, the difference between pain scores may not be comparable in scaling the intensity of pain \cite{Adeboye2021}. Finally, its verbal component is a limitation for non-verbal patients who may not be able to describe their pain using this method.

Another traditional method for assessing pain is the Visual Analogue Scale. This scale represents a continuous range of values, and it mainly uses a horizontal line measuring exactly 100 mm (10 cm). The patient is asked to make a mark on the line according to their level of pain, then, the distance of the mark is measured and recorded in millimetres or centimetres \cite{Bielewicz2022}. However, even though it’s non-verbal, a minimum level of motor abilities is necessary to correctly use the VAS, which makes it inadequate for scaling pain in some patients with motor impairment, for example. This method, although it seems to be more detailed in values than the NPS, is still very subjective, not allowing for an accurate pain description and comparison. This method can also be used for assessing other symptoms, however. For example, in the dataset used in this project, it was used to assess the level of anxiety, happiness, fear and stress that the participants felt, on a scale of 0-100\%, and the arousal and valence states in a -5 to 5 scale \cite{Alves2024}.

Other similar scales, that also depend on a patient's perception of their own pain, end up also being subjective \cite{Adeboye2021}\cite{Robinson2024}. Due to this, researchers have attempted to find objective ways to describe pain using physiological signals.


\section{Pain Description}

In this project, ECG was the physiological signal chosen for analysis.

