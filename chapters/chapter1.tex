\chapter{Introduction}%
\label{chapter:introduction}

\begin{introduction}
Face your life, its pain, its pleasure, leave no path untaken. - Neil Gaiman
\end{introduction}


\section{Motivation}
Pain is unpleasant. No one enjoys experiencing it, whether it's the muscle soreness from working out or the more severe pain from a broken bone. In fact, it might even represent a significant burden on a person's quality of life. The unpleasant feeling it provokes may result in functional disability or even a change of behaviour, making an individual feel anger and frustration \cite{Dirk2021}. However, pain serves a valuable purpose: it alerts us to potential dangers and helps us adapt to situations that could harm us, helping us avoid further injuries \cite{Coninx2021}. Nevertheless, when it interferes with the way people typically live their lives, it's important to find ways to attenuate it. This burden is particularly profound for patients suffering from chronic pain conditions, for whom discomfort becomes a persistent and debilitating element of life.

Attempts of measuring pain surged with the intent of improving quality of pain management \cite{Nugent2021}. For this reason, scales like the \ac{nps} and the \ac{vas} have been used since 1950 \cite{Bielewicz2022}. Although these are the most used methods, they're very subjective, depending on a patient's own perspective of pain. Moreover, some patients might undermine the severity of their pain or be unwilling to share their experience. For those with cognitive difficulties, these methods of verbal report may not even be appropriate, especially for non-verbal individuals \cite{Qin2022}.

The challenge of measuring pain makes prescribing pain medication a difficult job for healthcare providers since the only comparison they have is a patient's previous results. This means that the prescription can be excessive and the organism will grow accustomed to it quicker or, if the patient minimizes the severity, they will continue to experience pain as their tolerance increases. This highlights the importance of finding an objective way of describing pain.





\section{Work Objectives}









