\chapter{Introduction}%
\label{chapter:introduction}

\begin{introduction}
A short description of the chapter.

A memorable quote can also be used.
\end{introduction}


\section{Motivation}
Pain is unpleasant. No one enjoys experiencing it, whether it's the muscle soreness from working out or the more severe pain from a broken bone. However, pain serves a valuable purpose: it alerts us to potential dangers and helps us adapt to situations that could harm us, helping us avoid further injuries \cite{Coninx2021}. Nevertheless, when it interferes with the way people typically live their lives, it's important to find ways to attenuate it. This burden is particularly profound for patients suffering from chronic pain conditions, for whom discomfort becomes a persistent and debilitating element of life.

Attempts of measuring pain surged with the intent of improving quality of pain management \cite{Nugent2021}. For this reason, scales like the \ac{nrs} and the \ac{vas} have been used since 1950 \cite{Bielewicz2022}. Although these are the most used methods, they're very subjective, depending on a patient's own perspective of pain. Moreover, some patients might undermine the severity of their pain or be unwilling to share their experience. For those with cognitive difficulties, these methods of verbal report may not even be appropriate, especially for non-verbal individuals \cite{Qin2022}.

The challenge of measuring pain makes prescribing pain medication a difficult job for healthcare providers since the only comparison they have is a patient's previous results. This means that the prescription can be excessive, making the effects run out quicker, or if the patient minimizes the severity, they'll keep being in pain while their organism is growing accustomed to the medication.









Pain acontece quando? Não sei, mas está associado ao CNS - podemos usar HRV para descrever. Mas e ECG? features e tal. vamos nós fazer isso. pesquisar se mais alguém já fez isso


\section{Pain}

Pain, in its various forms, represents a significant burden on a person's quality of life. The unpleasant feeling it provokes may result in functional disability or even a change of behaviour, making an individual feel anger and frustration \cite{Dirk2021}. According to its duration, pain can be classified as acute or chronic. Acute pain is induced by the activation of nociceptor sensory neurons, which occurs in the presence of actual or potential damaging stimuli, such as intense heat or cold and excessive mechanical force, or due to inflammation \cite{Jayakar2021}. On the other hand, chronic pain is defined as lasting more than three months \cite{Raman2022} and can be classified into nociceptive, neuropathic or nociplastic pain. Nociceptive pain results from continuous stimuli associated with tissue injury, while neuropathic pain results from damage to the peripheral or central nervous system. Lastly, nociplastic pain is a broader term, that is applied to chronic pain when it can't be described by the other two terms \cite{Fitzcharles2021}.

The Numerical Rating Scale (\ac{nrs}) is one of the most widely used traditional methods for assessing pain. It typically involves asking patients to rate their pain on a scale from 0 to 10, where 0 represents no pain and 10 signifies the worst pain imaginable \cite{Nugent2021,Adeboye2021}. 

(No artigo deste Adeboye, falar do FPS - falar dos vários tipos de pain scale)
