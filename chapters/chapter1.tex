\chapter{Introduction}%
\label{chapter:introduction}

\begin{introduction}
A short description of the chapter.

A memorable quote can also be used.
\end{introduction}


\section{Motivation}
Pain is unpleasant. No one enjoys experiencing it, whether it's the muscle soreness from working out or the more severe pain from a broken bone. However, pain serves a valuable purpose: it alerts us to potential dangers and helps us adapt to situations that could harm us, helping us avoid further injuries \cite{Coninx2021}. Nevertheless, when it interferes with the way people typically live their lives, it's important to find ways to attenuate it. This burden is particularly profound for patients suffering from chronic pain conditions, for whom discomfort becomes a persistent and debilitating element of life.

Attempts of measuring pain surged with the intent of improving quality of pain management \cite{Nugent2021}. However, 

-> Some patients might be reluctant about sharing the experience of pain with their healthcare provider, or might minimize the severity of pain to avoid the deleterious effect of opioids[8]. Moreover, the verbal self-report might be inappropriate for nonverbal parents or those who are cognitively impaired. - by Jin Qin


Basically, sentir dor é uma merda. Seja porque estamos com dores musculares após um treino, ou numa situação mais intensa em que partimos uma parte do corpo, ninguém gosta de sentir dor (só masoquistas talvez). Ah até porque meio que nos impede de viver a nossa vida normalmente. Para pacientes com dor crónica então, coitados. Para os médicos, que têm de prescrever medicação aos seus pacientes, por vezes consoante a dor que eles sentem, é uma merda ainda maior uma vez que não têm métodos objetivos de classificar a dor dos pacientes. Por esta razão, acabam por usar a Numerical Pain Scale (NPS), em que os pacientes classificam a sua dor conforme uma escala de 0 a 10. Obviamente, isto não funciona devido às razões bue fixes que foram descritas no artigo q eu vi.



Problemas de não conseguirmos classificar dor: prescrição de medicamentos pode ser feita em excesso, então perdem efeito mais rápido; ou então dá-se pouco porque paciente se faz de forte, então ele continua só a sentir bue dores

Pain acontece quando? Não sei, mas está associado ao CNS - podemos usar HRV para descrever. Mas e ECG? features e tal. vamos nós fazer isso. pesquisar se mais alguém já fez isso


\section{Pain}

Pain, in its various forms, represents a significant burden on a person's quality of life. The unpleasant feeling it provokes may result in functional disability or even a change of behaviour, making an individual feel anger and frustration \cite{Dirk2021}. According to its duration, pain can be classified as acute or chronic. Acute pain is induced by the activation of nociceptor sensory neurons, which occurs in the presence of actual or potential damaging stimuli, such as intense heat or cold and excessive mechanical force, or due to inflammation \cite{Jayakar2021}. On the other hand, chronic pain is defined as lasting more than three months \cite{Raman2022} and can be classified into nociceptive, neuropathic or nociplastic pain. Nociceptive pain results from continuous stimuli associated with tissue injury, while neuropathic pain results from damage to the peripheral or central nervous system. Lastly, nociplastic pain is a broader term, that is applied to chronic pain when it can't be described by the other two terms \cite{Fitzcharles2021}.

The Numerical Rating Scale (\ac{nrs}) is one of the most widely used traditional methods for assessing pain. It typically involves asking patients to rate their pain on a scale from 0 to 10, where 0 represents no pain and 10 signifies the worst pain imaginable \cite{Nugent2021,Adeboye2021}. 

(No artigo deste Adeboye, falar do FPS - falar dos vários tipos de pain scale)
